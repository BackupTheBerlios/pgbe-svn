\documentclass[b5paper,10pt,dvips,fleqn,titlepage,twoside]{book}
%\pagestyle{headings}
\usepackage[utf8]{inputenc}
\usepackage{amsfonts}
\usepackage{multirow}
\usepackage[ngerman]{babel}
\usepackage[T1]{fontenc}
\pagestyle{headings}
\usepackage{rotating}
\usepackage{subfigure}
\usepackage{xcolor}

\usepackage[dvips]{hyperref}
\hypersetup{%
   pdfauthor=Michael Durrer,%
   pdfstartview=%
}

\author{Michael Durrer}
\title{Linux Game Development}
\date{2007-01-21}
\usepackage{makeidx}
\makeindex
\begin{document}

\begin{titlepage}
\begin{center}
\begin{huge} \textbf{Game \& Emulation Development on GNU Linux}\end{huge}
\newline
\newline
\begin{small}Written in \LaTeX by Michael Durrer\newline\end{small}
\end{center}
\footnotetext{ Last update on \today }
\end{titlepage}
\newpage
\tableofcontents
\setcounter{secnumdepth}{2}
\part{GNU Linux - Das fremde System}
\section{Vorwort}
Sie haben dieses Buch gekauft, von meiner Seite als PDF runtergeladen oder sonstwie erhalten mit einem bestimmten Hintergedanken und einer Erwartungshaltung. Und vielleicht haben sich auch schon einmal eine der folgenden Fragen an den Kopf geworfen:

\begin{flushleft}
\emph{
Wie programmiere ich ein Spiel oder eine Anwendung \underline{f\"{u}r Linux?}\newline Wo(mit) fange ich \"{u}berhaupt an?\newline Kann ich das \"{u}berhaupt?}\newline
\end{flushleft}

So ähnlich zumindest waren meine \"{u}berlegungen, als ich vor vielen Jahren angefangen habe, mich mit der Interna von Rechnern und der Digitaltechnik herumzuschlagen. Seien Sie also nicht beunruhigt, jeder ist sich anfangs unsicher. Und in Erinnerung an die Problematik dieses Beginns (der heute in der riesigen F\"{ü}lle an Informationen im Internet immerhin erleichtert worden ist), m\"{o}chte ich dieses Buch all denen zur Verf\"{u}gung stellen, die sich gerade inmitten jener Phase befinden und händeringend brauchbare Informationen suchen oder bereits diverse Grundlagen besitzen und darauf aufbauen m\"{o}chten.\newline
Vielleicht haben Sie bereits früher einmal versucht ein Spiel zu programmieren und es gelang Ihnen nicht, die Applikation fertigzustellen (übrigens eine der häufigsten Ursachen für den Abbruch eines Projektes: \textit{Motivationsverlust}). Möglicherweise sind Sie auch schlichtweg an einer Stelle in Ihrem Workflow angeeckt und haben keine Lösung für das Problem gefunden.
Gerade in solchen Fällen ist es mein Ziel, Ihnen mit diesem Buch lückenlos jede Stufe der praktischen Umsetzung von der Idee zur Applikation aufzuzeigen oder zumindest als Stütze dienen zu können.
Aus den verschiedenen Beweggründen und Ursachen, die jemanden zu diesem Buch geführt haben, lassen sich verschiedene Typen herauskristallisieren und auf einen gemeinsamen Nenner bringen: So mögen Einige bereits programmieren können in C/C++, Andere haben gar noch nie etwas entwickelt aber würden gerne den Einstieg finden und wieder Andere wollen sich schlichtweg weiterbilden, zum Beispiel über SDL und Emulatoren. Aus diesem Grund habe ich dieses Buch in mehrere Parts unterteilt, welche sich an verschiedene Niveaustufen richten und Neueinsteigern wie Quereinsteigern einen Einstieg in die Thematik bieten kann. Diese Parts sind:
\newline

\begin{itemize}
\item Hardware-Grundkenntnisse und Wissenswertes
\item GNU Linux - Das fremde System
\item SDL - Der Simple DirectMedia Layer
\item ANSI-C-Kurs für Anfänger
\item C++ Kurs (aufbauend auf den ANSI-C Kurs)
\item Die Grundlagen von Python
\item Emulation - Fremde Systeme simulieren
\end{itemize}\medskip 
Dabei ist der Schwerpunkt auf Linux gesetzt, doch habe ich auch durchgehend Wert auf Portabilit\"{a}t gelegt. Alle Programmbeispiele sollten auch unter Windows 2000 und höher laufen.\newline
Speziell in der Spielewelt w\"{a}re eine h\"{a}ufigere Verwendung der SDL angebracht, ist sie doch portabel auf alle g\"{a}ngigen Plattformen wie Windows Vista/XP, Linux oder Mac OS X.\newline
Portabel heisst in diesem Sinne, dass durch Verwendung von SDL f\"{u}r Steuer-, Eingabe-, Audio- und Videoger\"{a}te diese universal programmierbar werden. Eine einheitliche API (\textit{Application Programming Interface}) steuert die Ger\"{a}te nun an, der Programmierer sieht nur noch die API und deren Dokumentation. Spezifische Eigenheiten f\"{u}r jeweilige Ger\"{a}te und Betriebssysteme, z.B. verschiedene Grafikkarten, \"{u}bernimmt nun die SDL-Schicht komplett und \"{u}bersetzt sie in die Sprache der jeweiligen Betriebssysteme und Hardware um.\newline

Zus\"{a}tzlich habe ich mich f\"{u}r dieses Buch auf 3 Programmiersprachen festgelegt: \textbf{C, C++ \& Python}.\newline C/C++ ist quasi ein Industrie-Standard und wird seit Jahren in der Applikationsentwicklung da eingesetzt, wo man schnelle und \"{u}bersichtliche Anwendungen ben\"{o}tigt, zu diesen geh\"{o}ren auch grafiklastige Applikationen. Seit einigen Jahren sind aber die Prozessoren mittlerweile an einem Punkt angelangt, wo sich die Entwicklung nicht mehr so stark an Geschwindigkeit multipliziert wie fr\"{u}her.

Die meisten Rechner sind heutzutage schon bei einer Geschwindigkeit angelangt, mit der die meisten Spiele auf dem Markt lauff\"{a}hig sind, zudem wird immer mehr Rechenleistung auf die enorm leistugnsf\"{a}higen Grafikkarten ausgelagert, die auch auf langsameren Rechnern eine gute Grafikleistung hervorzaubern k\"{o}nnen und nicht mehr zwingend eine 'schnelle Programmiersprache' ben\"{o}tigen.

An dieser Stelle springt Python ein: Python ist eine skriptbasierte Sprache, die ebenfalls objektorientiert aufgebaut ist und somit eine exzellente \"{u}bersicht bietet. Desweiteren ist sie sehr einfach zu erlernen und kann ebenfalls mit SDL umgehen, es richtet sich daher eher an die Programmieranf\"{a}nger, was jedoch nicht heissen soll, dass man damit nicht genauso komplexe Applikationen bauen k\"{o}nnte.\newline Wie auch immer Ihre F\"{a}higkeiten und Ihr Wissen derzeit liegen, Ich hoffe Sie finden mit diesem Buch ein Themengebiet, dass Sie weiterbringt und Ihnen einen einfachen Einstieg in die Thematik erm\"{o}glicht.
Beachten Sie jedoch bitte, dass ich Ihnen nahelege, gute Vorkenntnisse in ANSI-C oder C++ mitzubringen, da ich den Grossteil dieses Buches doch darauf st\"{u}tze und auf jeden Fall fr\"{u}her oder sp\"{a}ter f\"{u}r professionelle Anwendungsentwicklung gelernt werden muss.

An den Teil \"{u}ber \textit{Emulation}, sollten sich nur echte Profis ranwagen, welche C/C++ bereits in- und ausw\"{a}ndig kennen. Auch wenn vielleicht nicht alles beim ersten Durchlesen verst\"{a}ndlich sein mag f\"{u}r Anf\"{a}nger, interessant ist es dennoch auf alle Fälle. Gleichzeiti wird Fach- und Hintergrundwissen vermittelt, welches unterstützend mitwirken kann beim Verst\"{a}ndnis Abläufe in der Programmierung.
Denn nur wer genau versteht, was sein Gerät spricht und tut, kann ihm genau befehligen, was und wie das Gerät es zu tun hat.

Einen Schwerpunkt habe ich beim Schreiben des Codes auf Übersichtlichkeit und einfachen Code, für Anfänger verständlich, gelegt, da es mir sehr wichtig war, Anfängern bereits die Möglichkeit zu bieten, komplexere Vorgänge wie Video- und Audioprogrammierung zwar nicht zwingend selber programmieren zu können, doch zumindest das Prinzip zu verstehen.

Neben dieser geballten Ladung an Information und Wissen, dass Sie in gebundener Form vor sich liegen haben, sollten Sie jedoch ein Punkt nie aus den Augen verlieren: Programmieren ist nicht nur lernen und arbeiten, sondern bereitet auch durchaus Entspannung und Spass! Und eben diese w\"{u}nsche ich Ihnen nun mit diesem Buch und entlasse Sie in die spannende Welt der Linuxprogrammierung...


Ihr Michael Durrer
\newpage
\chapter{Hardware-Grundkenntnisse und Wissenswertes}
In diesem Kapitel möchte ich auf einige Themen eingehen, die jeder (\emph{Linux-})Programmierer kennen oder zumindest einmal davon gehört haben sollte. Darunter fallen Sachen wie \emph{Aufbau von Bits \& Bytes, Berechnungen mit ODER-/UND-/EXODER-Tabellen, Ablauf von Programmen, Aufbau von Speicher und Hardware in einem PC-System} und vieles Weiteres. Ganz besonderen Wert lege ich auf emulationsrelevante Themen wie die \textbf{Struktur eines Mikroprozessors} oder die Wichtigkeit bei der Beachtung der \emph{Byte-Reihenfolge} beim Ablegen/Auslesen von Adressen und Daten im Speicher auf Systemen mit \textbf{Big- und Little-Endian}.

Zwar möchte ich nicht auf alle Grundthemen eingehen, da dies nun wirklich den Rahmen des Buches sprengen würde, doch zumindest einige wichtige Grundbegriffe und Theorien sollten in diesem Kapitel schon vermittelt werden.\newline
\newpage
\section{Bits and Bytes - Wie der Computer Zahlen liest und speichert}
Wir wir wissen, ist der Speicher bei Computern in verschiedene Einheiten unterteilt, wovon die kleinste als \textbf{Byte} bezeichnet wird, welche wiederum aus 8 \textbf{Bits}, welche jeweils den Zustand \emph{Wahr/True} oder \emph{Unwahr/False}, also 1 und 0, besitzen können.

Ein Byte verwendet das binäre Zahlensystem und kann mit 8 Bits/Stellen insgesamt 256 verschiedene Zustände und davon nur einen gleichzeitig  besitzen, d.h. von 0-255 zählen. Nach 255 (binär: 1111 1111) springt die Zahl wieder auf 0 (binär: 0000 0000).

Binärzahlen kann man zu einer Dezimalzahl umrechnen, indem man aus untenstehender Tabelle die Zahlen oberhalb der Spalte, wo unten eine 1 steht, alle zusammenzählt. In der linken Spalte stehen einige beliebige Zahlen. Versuchen Sie es selber mit einigen Zahlen, indem Sie von einer Binärzahl zu Dezimal und von einer Dezimalzahl zu Binär umrechnen.


\begin{table}[h]
\caption{Eine Rechentabelle für die Umrechnung von Binär- zu Hexadezimalzahlen}
\begin{tabular}{|l|c|c|c|c|c|c|c|r|}\hline
Zahl & 128 & 64 & 32 & 16 & 8 & 4 & 2 & 1 \\\hline\hline
255 & 1 & 1 & 1 & 1 & 1 & 1 & 1 & 1\\\hline
127 & 0 & 1 & 1 & 1 & 1 & 1 & 1 & 1\\\hline
56 & 0 & 0 & 1 & 1 & 1 & 0 & 0 & 0\\\hline
0 & 0 & 0 & 0 & 0 & 0 & 0 & 0 & 0\\\hline
\end{tabular}
\end{table}

Wie wir in der Tabelle sehen, hat ein 1 je weiter es links steht einen umso höheren Wert. Dies geht immer so weiter, denn wenn wir beispielsweise eine Zahl speichern wollen, die höher als 255 ist oder mehr als 256 Zustände benötigt, dann brauchen wir bereits 2 Bytes. Mit 2 Bytes lassen sich bereits Zahlen \newline bis zu  ($2^{16}$ =)  65536 Zustände darstellen, respektive ($2^{16}-1$ =) 65535 Zahlen darstellen.

\oddsidemargin -0.6in
\begin{table}[h]
\caption{Eine Rechentabelle für 2 Byte-Zahlen}
\begin{tabular}{|l|c|c|c|c|c|c|c|c|c|c|c|c|c|c|c|r|}\hline
Bit Nr. & 15 & 14 & 13 & 12 & 11 & 10 & 9 & 8 & 7 & 6 & 5 & 4 & 3 & 2 & 1 & 0\\\hline
Zahl & 32768 & 16384 & 8192 & 4096 & 2048 & 1024 & 512 & 256 & 128 & 64 & 32 & 16 & 8 & 4 & 2 & 1\\\hline\hline
65535  & 1 & 1 & 1 & 1 & 1 & 1 & 1 & 1 & 1 & 1 & 1 & 1 & 1 & 1 & 1 & 1\\\hline
65534 &1 & 1 & 1 & 1 & 1 & 1 & 1 & 1 & 0 & 1 & 1 & 1 & 1 & 1 & 1 & 0\\\hline
32768 & 1 & 0 & 0 & 0 & 0 & 0 & 0 & 0 & 0 & 0 & 0 & 0 & 0 & 0 & 0 & 0\\\hline
65280 & 1 & 1 & 1 & 1 & 1 & 1 & 1 & 1 & 0 & 0 & 0 & 0 & 0 & 0 & 0 & 0\\\hline
\end{tabular}
\end{table}

Nehmen wir nun zum Beispiel die Zahl 65535 in hexadezimaler Kodierung: \$FFFF
Jedes F-Doppel steht für ein Byte; FF ist die höchste Zahl, die ein Byte annehmen kann, bzw. in hexadezimaler Kodierung mit einem Byte darstellbar ist: 255. Da die beiden Bytes nun aneinandergehängt wurden, haben die Bits des Bytes, welches die Bits 8-15 darstellt, eine höhere Wertung und es lassen sich so Zahlen bis 65535 darstellen.
\newpage
\section{Big- and Little-Endianness - Byte-Strukturierung im Speicher}
Nun könnte man eigentlich annehmen, dass die Zahl \$FF F0 = 65520 = 11111111 11110000 eigentlich folgendermassen irgendwo im Speicher stehen müsste:\newline
\begin{center}
\begin{tabular}{|l|c|r|}\hline
Speicheradresse & \$2000 & \$2001 \\\hline
\$FFF0 im Speicher & FF & F0\\\hline
\end{tabular}
\end{center}


Leider ist dem jedoch nicht (immer!) so. In den 70'er Jahren gab es bei der Entwicklung der ersten Mikroprozessoren verschiedene Hersteller, die die Mikroprozessoren auf unterschiedliche Art \& Weise bauten, was die Assembler-Sprache anging als auch die Methode des Ein- und Auslesens von Adressen und Daten im Speicher.\newline
So ergab es sich, dass Intel seit dem 8086er bis heute zu den x86er Modellen aufwärts den \textbf{Little-Endian-Standard} verwendet, während viele andere Prozessorhersteller wie z.B. \emph{Motorola} auf den \textbf{Big-Endian-Standard} setzten. Gute Beispiele dafür sind die \emph{PowerPC-Prozessoren} in den \emph{Apple-Rechnern} oder in den \emph{Amigas}.

Worin unterscheiden sich nun diese beiden Ansätze? Solange der Prozessor von einer Speicherstelle nur ein Byte auslesen muss, ist das noch nicht problematisch, da ein Byte immer auf dieselbe Art und Weise interpretiert wird, die Bits selber wechseln die Position ja nicht. Setzt man jedoch die Reihenfolge der Bytes um, so kehren sich natürlich die Zahlen um und dies führt natürlich zu enormen Problemen, wenn dies nicht rechtzeitig erkannt wird und der Prozessor richtig programmiert wird, im entsprechenden Endian-Stil.\newline

Um die Unterschiede zu verdeutlichen, habe ich ein kleines Beispiel vorbereitet:\newline

\begin{center}
\begin{tabular}{|p{2in}|c|r|}\hline
Speicheradresse & \$2000 & \$2001 \\\hline
\$FFF0 im Speicher, wie es bei Little-Endian abgelegt ist & F0 & FF\\\hline
\$FFF0 im Speicher,wie es bei Big-Endian abgelegt ist & FF & F0\\\hline
\end{tabular}
\end{center}
\newline
Wir sehen, dass bei Little-Endian der Wert nun umgekehrt als wie wir ihn von links nach rechts lesen würden im Speicher steht. Darauf weist auch der Name \emph{Little-Endian} hin: Die niederwertigeren Bits zuerst (also links) nach rechts, wo am Ende das höchstwertige Bit steht. Bei \emph{Big-Endian} ist es genau umgekehrt: Das Bit mit der grössten Wertigkeit kommt zuerst und liest sich, von links nach rechts genau gleich wie die Adresse \$FFF0, wie wir es normalerweise lesen oder schreiben würden.
\subsection{Anwendungszwecke}
Vielleicht fragen Sie sich, weswegen wir das wissen müssen, da wir in C/C++ ja bei normalen Programmen sowieso nie damit konfrontiert werden, da die Hinterlegung im Speicher ja C/C++, bzw. der Compiler für uns übernimmt. Dies hat einen einfachen Grund: Zum Beispiel bei der Netzwerkprogrammierung, werden die Datenpakete auch teilweise im Big-Endian-Format umhergeschickt, so dass wir natürlich wissen müssen am Ziel-Rechner, ob unser Computer met dem Netzwerk-Endian-Format kompatibel ist oder nicht und wie wir diese Datenpakete auspacken und wieder zusammenbauen müssen.
Ein weiterer Grund ist die Emulation von fremden Systemen/Mikroprozessoren, wie ich es im letzten Teil dieses Buches beschreibe. Dort werde ich einen Little-Endian-Rechner, den Intel 8080, emulieren und muss dementsprechend wissen, wie im emulierten Speicher der Prozessor seine Adressen hinterlegt, damit er die Daten richtig interpretieren kann.\\

\chapter{Programmieren unter Linux}


\part{SDL - Der Simple DirectMedia Layer}
\label{part:SDL}
\chapter{SDL-Initialisierung}
\section{Einige Hintergrundinformationen...}
SDL steht, wie schon im Titel erw\"{a}hnt, f\"{u}r die Simple DirectMedia Layer Library. Der Name sagt uns schon, dass uns SDL einen \textit{simplen} und \textit{direkten} Zugriff auf Medien, bzw. Multimedia-Hardware bieten soll.

	Sie wurde entwickelt von Sam Latinga, w\"{a}hrend er bei Loki Software, bekannt f\"{u}r ihre Linux-Portierungen von Windows-Spielen, als leitender Programmierer angestellt war.
SDL diente als Basis f\"{u}r die Portierung vieler Windows-Spiele, darunter \emph{Civilization: Call to Power}und \emph{Descent 3}, um nur Einige zu nennen.\\
Dank der vielen F\"{a}higkeiten der Bibliothek und ihren OpenGL-Erweiterungen hat, vorallem aber auch dank der LGPL-Lizenzierung (mehr dazu im Kapitel \textbf{Open Source Lizenzen - Was ist das eigentlich genau?}), haben zu einer enormen Verbreitung dieser Entwicklungsbibliothek gef\"{u}hrt. Mittlerweile entwickeln ein ganzer Haufen an professionellen als auch private Entwickler an SDL weiter und sorgen daf\"{u}r, dass auch weiterhin aktuelle Technologien leicht ansprechbar bleiben \"{u}ber ein vereinfachtes Interface.

Durch die grosse Verbreitung ergibt sich noch ein besonders interessanter Vorteil: Die Plattformunabh\"{a}ngigkeit. Mittlerweile unterst\"{u}tzt SDL mehr als nur alle g\"{a}ngigen Betriebssysteme wie Linux, Windows und Mac OS X sowie die meisten Hochsprachen wie C/C++ sondern auch viele Nischen-Betriebssysteme und -Plattformen. Unter Anderem AmigaOS, SEGA Dreamcast, Microsoft XBox, Sony Playstation u.v.m.\newline Ganz im Gegensatz zu DirectX von Microsoft, welches ja alles Andere als cross-platform-f\"{a}hig ist...

Unter Windows hat SDL zudem einen kleinen (selbst unverschuldeten) Haken: Microsoft l\"{a}sst den Zugriff auf die Multimedia-Hardware, insbesondere Grafikkarten, nur \"{u}ber ihr eigenes Entwicklungs-Kit zu, n\"{a}mlich DirectX. Somit kann auch die SDL-Schicht nur auf die DirectX-Schicht aufbauen und ist somit gezwungenermassen leicht langsamer unter Windows, wenn man das Programm mit einem Kompilat auf dem gleichen Rechner, jedoch unter einem anderen Betriebssystem testet. Dieser f\"{a}llt jedoch nicht allzustark ins Gewicht und seien wir doch ehrlich:\newline Uns ist die einfache Programmierung und das Erreichen eines gr\"{o}sseren Zielpublikums durch Cross-Platform-Kompatibilit\"{a}t ein paar Frames pro Sekunde wert, oder? 
\newpage
\section{Installation}
Zuerst brauchen wir nat\"{u}rlich eine saubere SDL-Installation f\"{u}r unseren Compiler (in unserem Fall GCC).
\newpage
\subsection{Header-Dateien}
Um SDL benutzen zu k\"{o}nnen, m\"{u}ssen wir nat\"{u}rlich in unserer Hauptdatei (z.B. \emph{main.c} oder \emph{main.cpp}) die SDL.h Header-Datei inkludieren:
\begin{quote}
	\#include <SDL.h>
\end{quote}
Nun stehen uns alle Funktionen und Prozeduren der SDL-Welt zur Verf\"{u}gung! Willkommen in SDL!
Als n\"{a}chsten Schritt m\"{u}ssen wir einen Modus initialisieren, zum Beispiel um Video-spezifische Sachen darzustellen den Video-Modus.


\subsubsection{SDL-Erweiterungen}
Es gibt für SDL noch eine ganze Reihe an Erweiterungen, die allesamt auf der normalen SDL-Bibliothek aufbauen.
Eine kleine Auswahl:\\
\begin{table}[h]
\caption{Verschiedene hilfreiche SDL-Erweiterungen und Variationen}
\begin{tabular}{|l|p{2in}|r|}
\hline
\textbf{\emph{SDL-Erweiterung}} & \textbf{\emph{Beschreibung}} & \textbf{\emph{Hinweise}}\\\hline
SDL\_Image & Bild-Manipulation und diverse Grafik-Funktionen, liest versch. Bildformate ein& Win/Linux/OSX\\\hline
SDL\_Mixer & Audio-Kanäle mischen und Musikdateien abspielen (MP3, MIDI, MOD,...) & \\\hline
SDL\_Net & Netzwerk-Support für SDL, baut TCP/IP-Verbindungen auf, ideal für Spiele& \\\hline
SDL\_TTF & TrueType Font-Unterstützung & \\\hline
\end{tabular}
\end{table}


Zu beachten ist, dass alle diese Erweiterungen SDL bereits voraussetzen.
Besonders empfehlenswert, worauf ich auch in diesem Teil des Buches noch eingehen werde, ist die \emph{SDL\_Mixer Bibliothek}, diese gewährleistet Zugriff auf alle möglichen Grafikdateien und hilft uns, sie zu laden und zu schreiben.\newline
Erhältlich sind die Bibliotheken (und viele andere Sachen) allesamt unter:\newline

\url{http://www.libsdl.org}
\newpage
\section{SDL Modi initialisieren}

SDL kann in mehreren Modi initialisiert werden, jedoch werden nicht immer alle ben\"{o}tigt. Daf\"{u}r kann man mehrere Modi kreuzen. So braucht man beispielsweise in einigen Programmen gar keinen Audio-Modus oder es wird bei selbstablaufenden Programmen nichteinmal Input/Eingabe ben\"{o}tigt. 

Deswegen kann man Ressourcen (und damit Leistung!) sparen, indem wir nur das initialisieren, was wir auch wirklich ben\"{o}tigen.

Als Allererstes m\"{u}ssen wir SDL selber initialisieren, wir nehmen ersteinmal den Video-Modus.
Daf\"{u}r ben\"{o}tigen wir den Befehl \emph{SDL\_Init()}:
\begin{verbatim}
	int SDL_Init(Uint32 flags);
\end{verbatim}

Wir ersehen daraus, dass wir sogenannte Flags \"{u}bergeben, hier nehmen wir den Video-Modus:\newline
\begin{verbatim}
SDL_Init(SDL_INIT_VIDEO); 
\end{verbatim}


Nun haben wir den Video-Modus initialisiert und k\"{o}nnen bereits andere Video-Modi initialisieren.
Es gibt einige solcher Flags, hier die komplette Liste:
\newline

\begin{tabular}{|l|r|}
\hline SDL\_INIT\_EVERYTHING & Alle Subsysteme gleichzeitig starten \\\hline
SDL\_INIT\_INIT & Timer initialisieren\\\hline
SDL\_INIT\_VIDEO & Video Subsystem initialisieren\\\hline
SDL\_INIT\_INPUT & Eingabeger\"{a}te (Joystick, Maus, Tastatur) initialisieren\\\hline
SDL\_INIT\_CDROM & CD-/DVD-ROM Subsystem initialisieren\\\hline
SDL\_INIT\_JOYSTICK & Joystick Subsystem initialisieren\\\hline
SDL\_INIT\_NOPARACHUTE & SDL fängt keine fatalen Signale mehr ab\\\hline
SDL\_INIT\_EVENTTHREAD & Event Manager wird in einem separaten Thread gestartet\\\hline
\end{tabular}\newline

Besonders hinzuweisen ist noch auf \emph{SDL\_INIT\_NOPARACHUTE}.
Der SDL-Parachute ("`Fallschirm"`) schützt vorzusagen vor einem Absturz, einer 'unsanften' Bruchlandung. Er fängt zuverlässig 
\subsection{Die ODER-Tabelle}
Doch zuvor wollen wir noch sehen, wie man mehrere Modi miteinander kombiniert, wir möchten den Input-Modus zus\"{a}tzlich! Dies bewerkstelligen wir, indem wir die Werte bitweise miteinander in einer ODER-Tabelle verkn\"{u}pfen. Die ODER-Tabelle verkn\"{u}pft Bits nach folgendem Schema:
\begin{center}
\begin{table}[h]
\caption[tab:ORtable]{ODER-Tabelle mit Eingangsvariablen A + B und Resultat C.}
\hspace*{1.5in}\begin{tabular}[t]{|p{1in}||r|}
\hline \emph{ A ODER B }& \emph{C}\\\hline\hline
0 ODER 0 & 0\\\hline
0 ODER 1 & 1\\\hline
1 ODER 0 & 1\\\hline
1 ODER 1 & 1\\\hline
\end{tabular}
\end{table}
\end{center}
Das heisst, wenn mindestens eine der beiden Werte, bzw. deren jeweilige Bits die miteinander verkn\"{u}pft werden, 1 ist, ist das Ergebnis (C) 1. So kann man gut Werte miteinander verkn\"{u}pfen, bzw. fehlende Bits auff\"{u}llen.\newline
Dazu müssen wir nun jedes Bit eines Wertes dem dazugehörigen Bit des anderen Wertes ODER-verknüpfen.
Das ODER-Zeichen ist in C \& C++ die Pipe: |\newline\\
Ein kleines Beispiel: \\
Wir wollen Wert A mit Wert B verkn\"{u}pfen; in der Tabelle steht das neue Resultat, dass wir in C dann erhalten.\newline In der Tabelle sind alle ODER-Fälle enthalten, es ist also transparent, wie der Prozess abläuft.
\\\\
\hspace*{0.7in}\begin{tabular}[h]{|l|c|c|r|}
\hline Zahlensystem & Binär & Dezimal & Hexadezimal \\\hline\hline
Wert A & 0111 0001 & 71 & 113 \\
Wert B & 0000 1111 & 0F &  15 \\ \hline
Wert C & 0111 1111 & 80 & 128 \\ \hline
\end{tabular}
\newpage
\part{ANSI-C-Kurs für Anfänger}
\label{part:ansic}
\chapter{Hintergrundwissen}
\section{Entwicklung und Geschichte von C}
C gehört zu den sogenannten Hochsprachen und wurde in den 1970'er Jahren entwickelt von \emph{Ken Thompson} und \emph{Dennis Ritchie} in den Bell Labs. Ursprünglich, als eine der ersten Hochsprachen, gab es \emph{ALGOL 60} die etwa 1960 entwickelt worden war von einem internationalen Komitee.

Daraus wurde dann \emph{CPL} (\textbf{C}ombined \textbf{P}rogramming \textbf{L}anguage) geschaffen, woraus dann wiederum \emph{BCPL} (für \textbf{B}asic \textbf{CPL}) entstand. 1970 entstand dann einer der direkten Vorgänger von dem, was später als (ANSI-)C zu weltweiter Verbreitung finden sollte: Simpel und schlicht \textbf{B} wurde es genannt.

Vielen ist bekannt, dass wir die Entwicklung der Programmiersprache \emph{C} den Bell Laboratories zu verdanken haben, welche im Zuge Ihrer Planung und Entwicklung von \emph{UNIX} das System später sogar in C schrieben.
Denn zunächst entstand durch die beiden Entwickler \emph{Ken Thompson} und \emph{Dennis Ritchie} am \emph{MIT} das \emph{Betriebssystem UNIX}, und zwar noch in \textbf{Assembler}\footnotetext{Maschinensprache: Für den Computer direkt verständliche Codes (\emph{Mnemonics}).}.
\newline


Da Assembler nicht gerade leicht gerade gut lesbar und verständlich ist aufgrund ihrer Struktur und der verwendeten Zahlensysteme, reifte in den beiden Entwicklern der Wunsch, eine einfache wie schnelle Programmiersprache zu entwickeln, welche nicht bloss auf einem Rechner lauffähig sein sollte, sondern auf jedem Rechner lauffähig sein sollte ohne grosse Anpassungen am Quelltext vornehmen zu müssen.\newline
Der grosse Nachteil in Assembler lag nämlich besonders darin, dass für nahezu jeden Prozessor wieder ein teilweise komplett anderer Befehlssatz offeriert wurde, hinzu kamen andere Eigenheiten wie verschiedene \textit{Byte-Reihenfolgen}\footnotetext{Lesen Sie dazu bitte das Kapitel $\rightarrow$\emph{Little Endian vs. Big Endian - Byte-Reihenfolgen und ihre Auswirkungen} und systemspezifische Arten der Speichernutzung.}

Da nun \emph{UNIX} eben aus oben genannten Gründen nicht auf allen Systemen lauffähig war, beschloss der junge Wissenschaftler \emph{Dennis Ritchie}, eine Sprache zu entwickeln, die für solche performantenwie portablen Applikationen, bzw. deren Entwicklung, geeignet war und schuf auf einer \emph{PDP-11}, ein damals weitverbreiteter Grossrechner, im Jahr 1971 auf Basis der Programmiersprachen \textbf{BCPL} und \textbf{B} endlich die erste Version von \emph{\textbf{C}}!

Schon bald darauf folgte die Feuertaufe im Jahre 1972 für C: UNIX wurde nun von \emph{Assembler} in \emph{C} umgeschrieben und für den PDP-11 veröffentlicht.\newline
In den nächsten Jahrzehnten sollte es sich auf der ganzen Welt rasant verbreiten und weiterentwickeln . Ursprünglich nur für Grossrechner und die Industrie gedacht, verbreitete es sich in den 80'er Jahren ebenfalls dank der ersten Umsetzungen für Intel-Rechner auf Home-Computer von Privatleuten, welche dank UNIX die Kapazität ihrer Heimrechner immer mehr ausnutzen wollten und erstmals konnten. Eine der erfolgreichsten, unixoiden (sprich: \emph{UNIX}-ähnlichen) Betriebssysteme hat weltweite Verbreitung auf alle möglichen Arten von Prozessoren und Rechnern gefunden und ist uns heute bekannt unter dem Namen \textbf{Linux}!

Ganz im Sinne der Entwickler haben sich \emph{UNIX}, bzw. heute vermehrt auch \emph{Linux} weltweit überwiegend auf Servern, aber auch auf Client-Rechnern, verbreitet und wurden auf alle möglichen Plattformen portiert. Nicht zuletzt verdankt man diesen grossen Erfolg  der \textbf{GPL}-Lizenzierung \footnote{Mehr Informationen und Hinweise hierzu finden Sie im Kapitel "'Programmieren unter Linux"`}, die einen grossen Teil zur Popularität und öffentlichen Beteiligung beigetragen hat durch Offenlegung der Quelltexte.

Und genau wie sein Urahn UNIX wurde Linux zu einem überwiegenden Teil in \emph{C} geschrieben.
Die meisten Distributionen\footnote{Distributionen sind Zusammenstellungen von Linux-Software mit Linux als Kern(el). Sie werden von kommerziellen Betreibern und der privaten Linux-Community gepflegt und gewartet.} liefern heute standardmässig einen C-Compiler (üblicherweise den \emph{gcc} $\rightarrow$ \emph{GNU C Compiler}) und den Quelltext zur Distribution mit; so wird dem Benutzer gleich von Anfang an eine Entwicklungsumgebung zur Verfügung gestellt, die ihm erlaubt, das System nach seinen Wünschen anzupassen oder zu verändern, bzw. sogar Fehler zu berichtigen und Features zu programmieren, welche er in die Distribution zurückfliessen lässt und somit zur ständigen Weiterentwicklung beiträgt.

Um nocheinmal zu C zurückzukommen: Durch die Vervielfältigung des sich ständig in Wandlung befindlichen Betriebssystems UNIX / Linux, gab es während den 70'er und 80'er Jahren keine eindeutige Definition, welchen Sprachumfang C nun genau umfasst.

So besitzt C zum Beispiel kaum eigene Funktionen, liefert jedoch im Standardumfang div. Funktionen wie z.B. \textbf{printf()} zur flexiblen Textausgabe mit, welches wohl jedem von "'Hello World"`-Programmen bekannt ist.
Um einer weiteren wilden Verbreitung Einhalt zu gebieten, wurde vom \textbf{ANSI} (\emph{\textbf{A}merican \textbf{N}ational \textbf{S}tandard \textbf{I}nstitute}), einem Komitee zur Festlegung und Festigung von Standards in vielen verschiedenen Gegenständen und Technologien des alltäglichen Gebrauchs. Diese dienen zur besseren Interkommunikation und Zusammenarbeit zwischen Institutionen, Firmen und letztendlich Menschen und optimieren diese.

1988 gab es somit die erste Sprachbeschreibung des ANSI und ein Jahr später erreichte eben diese Sprachbeschreibung den offiziellen Status eines ANSI-Standards und wird seither als \textbf{\emph{ANSI-C}} bezeichnet \lbrack wie in diesem Buch natürlich auch.\rbrack
\newline
\section{Über ANSI-C und C++}
Wie schon früher angedeutet und hingewiesen, ist \textbf{ANSI-C} eine flexible, portable Programmiersprache, die auf fast allen gängigen aber auch selteneren Betriebssystemen zur Verfügung steht und untereinander weitestgehend kompatibel sind, da die Sprache den \emph{ANSI-Standard} von C hält und somit keine allzugrossen Differenzen bei der Funktionalität entstehen.
Mit dem Erlernen von C investieren Sie in ein zukunftssicheres Wissen, welches auch in 20 Jahren in den Prinzipien noch seine Gültigkeit haben wird; desweiteren wird man ihre heute in \emph{ANSI-C} entwickelten Programme ebenfalls noch viele Jahre lang lauffähig übersetzen können.

Durch die hohe Verbreitung von \emph{C und C++} ist dieses Wissen in kommerzieller als auch, sogar insbesondere, in non-kommerzieller (\emph{Open-Source, Hobby-Entwickler}) Hinsicht hochgefragt.

Was mit Linux, UNIX und Mac OS X (\emph{Objective-C}), eignen sich diese Sprachen perfekt zur Systemprogrammierung. Darüberhinaus werden die C-Sprachen auch noch eingesetzt für Treiberentwicklung und für die Anwendungsentwicklung auf eingebetteten Systemen/Mikrocontrollern.

Die meisten der heutigen kommerziellen Anwendungen werden in \emph{C oder C++} geschrieben. Vermehrt kommt seit einigen Jahren auch \emph{Java} zum Einsatz, da im Gegensatz zu C/C++ die Quelltexte von Java nur ein einziges einmal kompiliert werden müssen und Anschluss auf jedem System, welches über eine \textbf{JVM}\footnote{Java Virtual Machine - Eine virtuelle System-Umgebung, in der Java-Anwendungen ablaufen} verfügt in der Lage ist, die Programme laufen zu lassen.

Es ist heutzutage offensichtlich, dass immer mehr Wert darauf gelegt wird, dass Programme leicht auf mehreren Plattformen lauffähig sind, und, wie schon erwähnt, bieten C und C++ genau diese Möglichkeit, den Quelltext ohne (grössere) Änderungen auf jeder Plattform zu übersetzen. Einer der weiteren Vorteile ist die geringe Speichernutzung, bzw. das optimale Speichermanagement, worin besonders C++ punkten kann. Nach C/C++ sind meistens nur noch Assembler-Anwendungen kleiner, welche jedoch nicht annähernd über dieselbe Flexibilität verfügen.

Um das Augenmerk auch noch ein wenig auf \emph{C++} zu lenken: C++ verfügt gegenüber C viele Vorteile. C++ selber ist eine \emph{Obermenge} von C, das heisst: Alle Elemente von C sind in C++ auch enthalten; darüberhinaus besitzt C++ jedoch noch viele weitere Fähigkeiten, die in der modernen Software-Entwicklung nicht mehr wegzudenken wären.

Dazu gehören unter Anderem:\newline

\begin{itemize}
 \item Objektorientierung: Bietet eine bessere Ordnung und Verwaltung von Projekten.
\item Die Speicherverwaltung ist mit \emph{new} und \emph{delete} weitaus absturzsicherer, einfacher und sauberer geworden. Vorallem ist die Allokation von Speicher nun Hauptbestandteil von C++ selber.
\item Die Objektorientierung bringt \emph{Vererbung} mit sich; das bedeutet, dass Klassen nun (in C bekannt als Funktionen:)Methoden und (Variablen:)Attribute von anderen Klassen erben können, während man in C noch komplett neue Strukturen und Funktionen schreiben musste.
\item Das Stream-Konzept: Daten werden in Form von Datenströmen, Streams, gelesen und geschrieben
\item Konstruktoren und Destruktoren: Beim Erzeugen und Löschen einer Instanz einer Klasse, werden selbstdefinierte Funktionen, den Konstruktor und den Destruktor, ausgeführt. Ideal um z.B. dazugehörigen Speicher gleich mitzulöschen, damit er nicht im Speicher hängenbleibt.
\item Polymosphirmus: Darauf gehe ich in C++-Kurs näher ein, da das hier sonst den Rahmen sprengt.
\end{itemize}

\section{Das erste Programm}
Für Ihr erstes C-Programm, können Sie untenstehenden Quelltext verwenden; diese Anwendung liest über die \textbf{Standardeingabe (STDIN)}, in Ihrem Fall die Tastatur, einen String ein. Jener String darf maximal 15 Zeichen betragen, da nicht mehr Speicher alloziiert worden ist, und wird nach der Eingabe mit Return in das Programm übergeben, welches diesen Satz anschliessend auf der \textbf{Standardausgabe (STDOUT)}, dem Bildschirm, ausgegeben.
\newpage
\begin{verbatim}
/* ReadString.c - Ein Programm, welches einen String einliest und über den Bildschirm ausgibt.
*/
\newlabel

#include <stdio.h> // Einbinden der Standard-Ein-/Ausgabe-Funktionen.

int main (int argc, char *argv[])
{
	char stringPhrase[16];  // String aus 15 Zeichen + 1 Zeichen für das Terminierungssymbol ('\0)
	scanf("%s",&stringPhrase[0]); // Einlesen des Strings von Tastatur
	printf("Ihre Eingabe lautete: %s",stringPhrase); // Eingabe ausgeben auf Bildschirm
	return(0); // Programm erfolgreich ausgeührt, Gebe Wert zurück
}
\end{verbatim}
\part{Der C++-Kurs}
\part{Emulation - Fremde Systeme simulieren}
\chapter{Von Emulatoren und Nostalgikern}
\section{Definition}
Bestimmt haben Sie schonmal von \textbf{Emulatoren} gehört, jenen Anwendungen, die auf Ihren PC alte Konsolen oder Computer, realitätsgetreu in Ihrer (Re)Aktion, herbeizaubern und es Ihnen ermöglichen, alte Applikationen und Spielanwendungen wieder zu benutzen und zu erleben. Doch was sind Emulatoren genau per Definition? Und was passiert da genau, was macht ein Emulator?\newline

Ersteinmal möchten wir dem Wort auf den Grund gehen: \emph{Emulation]} leitet sich vom lateinischen Wort für nachahmen ab, welches \emph{aemulari} lautet.
In der EDV wird das funktionelle Nachbilden eines Systems auf / durch einem/eines Anderen bezeichnet. Ein Emulator ist folglich ein System, welches ein anderes System nachahmt und komplett gleich reagiert auf Eingabe \& Verarbeitung der eingeführten Daten.

Das nachbildende System erhält dieselben Daten, führt dieselben Programme und \texttt{muss} die gleichen Ergebnisse erhalten wie das Originalsystem. Durch die sich immer schneller entwickelnde EDV-Technik der letzten Jahrzehnte kamen mitte der 90'er die ersten Software-Emulatoren. Moment mal, Software-Emulatoren? Wenn es Software-Emulatoren gibt, muss es auch Hardware-Emulatoren geben, und jene gab es natürlich: Zunächst in den 60'ern und 70'ern. Damals kamen Geräte in den Handel, die zum Beispiel kompatibel waren mit dem damals sehr populären Atari 2600, sie bildeten sein Verhalten elektronisch nach mittels anderer Hardware. Damals war die Hardware noch zu langsam, um soetwas über die Software zu realisieren.

Ein Nachteil dieser ganzen Geschichte liegt darin, dass ein Emulator nie 100\% gleich agieren kann wie das Original, auch wenn die Abweichungen nahezu irrelevant und nur minim sind. Somit bleiben gewisse Anwendungen / Spiele Nostalgikern und Spiele-Freaks heutzutage immer noch verschlossen vor der Verwendung auf einem Fremdsystem. Zugegebenermassen hat aber die Qualität bei Emulatoren mittlerweile dermassen zugenommen, dass man gute Chancen hat, für seine bevorzugte Applikation ein passendes Programm zu finden, welches fähig ist, die Applikation fehlerfrei auszuführen.

\section{Anwendungszwecke von Emulatoren}
Für Emulatoren gibt es haufenweise Anwendungszwecke; bei einigen mag sich der Sinn zwar nicht unbedingt allen erschliessen, doch sind sie nicht umsonst sehr begehrt für jedes mögliche zu emulierende System.
So verwendet man Emulatoren unter Anderem für folgende Anwendungszwecke:
\newline
\begin{itemize}
\item Weiterverwenden von Software für ein anderes (Betriebs-)System nach der Migration in eine neue Umgebung.
\item Softwareentwicklung - Ein Programm kann ohne Gefahren für die eigene Hardware nach Belieben getestet und entwickelt werden im geschützten Rahmen der Emulation.
\item Freizeitvergnügen - Viele Leute spielen gerne im Emulator die Spiele von einst auf ihren modernen PCs.
\item Ergonomisches Arbeiten - Man kann seinen Arbeitsplatz an ein System verlagern, welches ergonomische Hardware bietet und das Zielsystem emulieren kann, so dass man nun ergonomisch arbeiten kann.
\item Applikationen dank einiger Zusatzfunktionen schneller ausführen lassen im Emulator, als wenn Sie auf dem echten Gerät laufen würde.
\end{itemize}
\section{Aufbau dieses Parts}
Sie sehen also, es gibt durchaus einige vernünftige und sinnvolle Zwecke zur Entwicklung und Verwendung eins Emulators.
Nicht zuletzt kann es eine sehr lehrreiche und interessante Sache sein, einen Emulator selbst zu entwickeln und zu verwenden. Ein Emulator ist rein vom programmiererischen Aspekt her eine sehr anspruchsvolle Arbeit und erfordert viel Verständnis für Aufbau und Abläufe eines Mikroprozessors und sonstiger Hardware. Daher habe ich auch entschlossen, diesen Part in meinem Buch zuletzt zu platzieren, da hier alles Vorwissen aus jedem vorherigen Part hier zusammenläuft und praktische Verwendung findet. Ohne tiefgreifende Kenntnisse der zu emulierenden Hard- und Software wird man jedoch seine Mühe haben, hier auch nur ansatzweise zum Erfolg zu kommen.

Diesem Buch beiliegend finden Sie eine komplette Dokumentation inkl. Quelltexte zu meinem Intel 8080 Emulator. In diesem Teil des Buches wird auch umfassend beschrieben, welche Entscheidungen und Design-Wege ich getroffen und gegangen bin, um den Emulator möglichst akkurat zu gestalten. Jeder Schritt soll für den Leser nachvollziehbar und für andere Systeme vom Prinzip her anwendbar sein.\newline
Es empfiehlt sich daher sattelfest in C++ und SDL zu sein und ein gutes Vorwissen mitzubringen im Bezug auf Emulation, Mikroprozessoren (Assembler), Aufbau des Arbeitsspeichers U.Ä.

\end{document}
